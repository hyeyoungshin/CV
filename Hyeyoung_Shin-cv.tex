%_______________________________________________
% @brief    LaTeX2e Resume for Hyeyoung Shin
\documentclass[margin,line]{resume}
\usepackage{url}
\usepackage[shortlabels]{enumitem}
%\usepackage{enumitem}

\usepackage{graphicx}

\usepackage[usenames,dvipsnames]{color}
\definecolor{light-gray}{gray}{0.70}
\definecolor{gray}{gray}{0.50}
\definecolor{dark-gray}{gray}{0.30}

\usepackage[colorlinks=true]{hyperref}

\hypersetup{
     citecolor    = gray,
     urlcolor=Blue
     %% linkcolor=Emerald
}


\usepackage{ifthen}
\usepackage[LabelsAligned]{currvita}     
\begin{document}

\newcommand{\Hawaii}{Hawai\kern.05em`\kern.05em\relax i}
\newcommand{\Manoa}{M\=anoa}
\name{{\Large Hyeyoung Shin} \hskip127mm {\large\sc Resum\'e}} %{\large\sc Curriculum Vit\ae}}

\begin{resume}

    %___________________________________________
    % Contact Information
    \section{\mysidestyle Contact\\Information}
     Naplavni 1772/2, Apt 53          \hfill tel: +420 603-535-554\\
     Prague 2                 \hfill url: \href{https://hyeyoungshin.github.io/}{www.hyeyoungshin.org}\\
     Czech Republic  \hfill             \hfill email:  \href{mailto:hyeyoungshinw@gmail.com}{hyeyoungshinw@gmail.com}

    %__________________________________________________________________________________
    % Research Interests
    \section{\mysidestyle Interests}

    Programming languages, type theory, and functional programming
    
    %________________________________________________________________________________
    % Education
    \section{\mysidestyle Education}

    \newcommand\mysmallskip{4pt}
    \newcommand\mymedskip{6pt}
    \newcommand\mybigskip{8pt}

    \textbf{Northeastern University}. \textsl{MSCS in Computer Science}\hfill 2017--2019\\
    Advisor: \href{http://www.ccs.neu.edu/home/amal/}{Professor Amal Ahmed}\\[\mymedskip]
    %
    \textbf{University of \Hawaii, \Manoa}. \textsl{Grad courses and exams in mathematics} \hfill 2016--2017\\  %\hfill Ames, IA\\
    \footnotesize{Graduate level: logic, recursion theory; undergraduate level: concurrent programming, topology} \\[\mymedskip]
    %
    \textbf{Iowa State University}. \textsl{Graded courses and exams in mathe and computer science} \hfill 2014--2016\\
    \footnotesize{Graduate: Programming languages, formal methods, computability; undergrad: OOP, data structures, algorithms, abstract algebra, intro to proofs, calculus} \\[\mymedskip]
    %
    \textbf{Kyeongpook National University}. \textsl{Bachelor of Arts, English Language and Literature} \hfill 2004--2009  %\hfill Daegu, South Korea\\
    

      

%______________________________________________________________________________________
\section{\mysidestyle Additional Training}

\newcommand\ssitem[5]{\href{#1}{#2} \hfill #3 \\ Topics: #4 \hfill #5}

\ssitem
{https://summer-school.racket-lang.org/2017/}
{The Racket School of Semantics and Languages}
{University of Utah}
{semantics and language design}
{10 July--14 July 2017}\\[\mymedskip]
\ssitem
{https://www.cs.uoregon.edu/research/summerschool/summer17}
{Oregon Programming Languages Summer School}
{University of Oregon}
{dependent, gradual, and substructural type systems}
{26 June--8 July 2017}\\[\mymedskip]
\ssitem
{http://www.cs.bham.ac.uk/~pbl/mgs2016/}
{Midlands Graduate School in Foundations of Computing Science}
{University of Birmingham}
{type theory, denotational semantics, category theory}
{11--15 April 2016}\\[\mymedskip]
\ssitem
{https://www.cs.uoregon.edu/research/summerschool/summer16/}
{Oregon Programming Languages Summer School}
{University of Oregon}
{type theory, logic, semantics, verification}
{20 June--2 July 2016}\\[\mymedskip]
\ssitem
{https://www.coursera.org/learn/progfun1}
{Functional Programming Principles in Scala}
{\'{E}cole Polytechnique F\'{e}d\'{e}rale de Lausanne}
{6-week online course with
\href{https://www.coursera.org/account/accomplishments/records/SRLRBNFMFW86}
     {verified certificate}}
{Grade Achieved: 94\%}

%_______________________________________________________________
% Work Experience
\section{\mysidestyle Professional\\Experience}
\textbf{Czech Technical University}.  \textsl{RA for the Signatr Project} \hfill 2019--2021\\
Advisor: \href{http://janvitek.org/}{Professor Jan Vitek} and \href{http://cs.uni-salzburg.at/~ck/}{Christoph Kirsch}\\[\mysmallskip]
\textbf{Czech Technical University}.  \textsl{TA for OOP design course by \href{https://fikovnik.net/}{Filip Krikava}} \hfill Fall 2020\\
\textbf{Iowa State University}.  \textsl{TA for data Structures course by \href{http://web.cs.iastate.edu/~jia/}{Yan-Bin Jia}} \hfill Fall 2015\\[\mysmallskip]
%\textsc{coms} 228: Introduction to Data Structures \\[\mymedskip]
\textbf{Gyeongsan Girls' High School}. \textsl{English Teacher}  \hfill 2009--2013

    
%%     %_______________________________________________________________    %         Volunteer work
%%     \section{\mysidestyle Volunteer\\Work}

%%     \textbf{InDaegu} \hfill 2011--2012\\
%%     \textsl{Korean-English translator:} 
%%     \href{http://www.in-daegu.com/}{InDaegu newspaper}.
%% %    \href{http://www.in-daegu.com/}{(in-daegu.com)}. 
    
%%         \textbf{Daegu Pockets}  \hfill 2010--2011\\
%%     \textsl{Korean-English translator:}
%%     \href{http://daegupockets.com/}{Daegu Pockets magazine}.
%% %    \href{http://daegupockets.com/}{(daegupockets.com).}
    
%%     \textbf{Daegu International Bodypainting Festival}  \hfill 2010\\
%%     \textsl{Korean-English translator:} DIBF Association.
    
%%     \textbf{Daegu International Musical Festival}   \hfill 2009\\
%%     \textsl{Korean-English translator:} DIMP Association.

%______________________________________________________________________________________
    \section{\mysidestyle Research}
    \textit{\href{https://github.com/PRL-PRG/signatr}{The Signatr Project}}: Build a system to infer function types dynamically\\
          {\footnotesize with Jan Vitek, Christoph Kirsch, Filip Krikava, and Yuan Cao}\\[\mymedskip]
    %
    \textit{\href{https://github.com/hyeyoungshin/popl19src/blob/master/popl19src.pdf}{A fully abstract compilation from
        a total to a partial language.}} H.~Shin (submitted to {\small POPL}'19SRC)\\ %%(TODO: add outcomes, contributions)

%______________________________________________________________________________________
    \section{\mysidestyle Programming Experience} 

    \textbf{R.} Building a tracer and database for function arguments and return values for a research project\\[\mymedskip]
    %
    \textbf{Racket.} Implemented an \href{https://github.com/hyeyoungshin/hy_eopl}{interpreter generator} parametrized by representations of environment and closure\\[\mymedskip]
    %
    \textbf{SML.} Implemented a compiler that compiles \href{https://www.cs.princeton.edu/~appel/modern/ml/}{Tiger language} to MIPS assembly.\\[\mymedskip]
    %
    \textbf{Other.} Scala, Java, Git, \LaTeX, 

    \section{\mysidestyle Awards} 
    %% \begin{itemize}
    %% \item Scholarship to attend
    %%   \href{https://www.cs.uoregon.edu/research/summerschool/summer16/}{OPLSS},
    %%   Oregon; 20 June--2 July, 2016.
    %% \item Scholarship to attend
    %%   \href{https://www.cis.upenn.edu/~sweirich/icfp-plmw15/}
    %%        {Programming     Languages Mentoring Workshop} at the\\
    %%        20th ACM SIGPLAN
    %%        {\bf \href{http://icfpconference.org/icfp2015/}{ICFP 2015}}
    %%        %% International Conference on Functional Programming, 
    %%        Vancouver, BC; 31 Aug--2 Sep, 2015.
    %% \item 
    %%   Scolarship to atten the
    %%   {\bf \href{http://conf.researchr.org/home/PLMW-2016}{POPL/PLMW 2016}}
    %%   Florida; 19 Jan--22 Jan, 2016.
    %% \end{itemize}

    Northeastern University Ph.D. Graduate Fellowship\hfill Boston, 2017--2018\\[\mymedskip]
    Scholarships to attend
    \href{https://www.cs.uoregon.edu/research/summerschool/summer17}
    {Oregon Programming Languages Summer Schools} \hfill %% ; 20 June--2 July, 
    Eugene, 2016, 2017\\[\mymedskip]
%     Scholarship to attend
%     \href{https://www.cs.uoregon.edu/research/summerschool/summer16}
%          {Oregon Programming Languages Summer School} \hfill %% ; 20 June--2 July, 
%     Eugene, 2016\\[\mymedskip]
    Scholarship to attend 
     \href{http://conf.researchr.org/home/PLMW-2016}{POPL
        Programming Languages Mentoring Workshop} \hfill St.~Petersburg, 2016\\[\mymedskip]
            Scholarship to attend
           \href{http://icfpconference.org/icfp2015/}{ICFP}
           \href{https://www.cis.upenn.edu/~sweirich/icfp-plmw15/}
                {Programming Languages Mentoring Workshop}
                %% at the\\    20th ACM SIGPLAN
         %% International Conference on Functional Programming, 
\hfill         Vancouver, %% , BC; 31 Aug--2 Sep, 
2015


   
    \section{\mysidestyle Leadership} 
     Organizer of \emph{PL Jr. Study \& Research Group}, Northeastern University \hfill 2018-2019

    \newpage
    %%\section{\mysidestyle Talks}
    %% \vspace{-3mm}
    %%  \href{https://github.com/hyeyoungshin/Talks/raw/master/HTT/htt-slides.pdf}
    %%       {Hoare Type Theory}, July 2016.


    %%\begin{center}
    %%RELEVANT COURSEWORK
    %%\end{center}
    %%______________________________________________________________________________________
    %%\section{\mysidestyle Computer Science}
    %%\begin{itemize}
%%    \item Graduate Programming Languages ({\small COMS} 542; grade: A)
%%    \item Graduate Theory of Computing ({\small COMS} 531; grade: A)
%%    \item Graduate Formal Methods ({\small COMS} 512; grade: A)
%%    \item Undergraduate Theory of Computing ({\small COMS} 331; grade: A-)
%%    \item Discrete Computational Structures ({\small COMS 330}; grade: A)
%%    \item Algorithms ({\small COMS} 311; grade: A)
%%    \item Data Structures ({\small COMS} 228; grade: A)
%%    \item Object-oriented Programming ({\small COMS} 227; grade: A)
%%    \end{itemize}

%%    %______________________________________________________________________________________
%%    \section{\mysidestyle Mathematics}
%%    \begin{itemize}
%%    \item Graduate Logic ({\small MATH} 654; grade: tbd)
%%    \item Topology ({\small MATH} 421; grade: tbd)
%%    \item Abstract Algebra ({\small MATH} 301; grade: A)
%%    \item Introduction to Proofs ({\small MATH} 201; grade: A)
%%    \item Calculus I, II ({\small MATH} 165, 166; grades: A, A)
    %% \item Calculus 2 ({\small MATH} 166; grade: A)
    %% \item Calculus 1 ({\small MATH} 165; grade: A)
%%    \end{itemize}



\end{resume}
\end{document}
















%%% Local Variables:
%%% mode: latex
%%% TeX-master: t
%%% End:
